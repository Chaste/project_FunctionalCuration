\documentclass[t,xcolor={usenames,dvipsnames}]{beamer}

\mode<presentation>
{
\usetheme{Frankfurt}%{Warsaw}
%\setbeamercovered{transparent}
%\setbeamercolor{background canvas}{bg=white}
}

% Delete these, if you do not want the table of contents to pop up at
% the beginning of each (sub)section:
%\AtBeginSubsection[]
%{
%  \begin{frame}<beamer>{Outline}
%    \tableofcontents[currentsection,currentsubsection]
%  \end{frame}
%}
%\AtBeginSection[]
%{
%  \begin{frame}<beamer>{Outline}
%    \tableofcontents[currentsection]
%  \end{frame}
%}

\usepackage[english]{babel}
\usepackage[latin1]{inputenc}
\usepackage{times}
\usepackage[T1]{fontenc}
\usepackage{verbatim}
\usepackage{url}
\usepackage{amsmath,amssymb}
\usepackage{comment}
\usepackage{hyperref}

% Author-date citations
\usepackage[authoryear,round]{natbib}
\let\cite=\citep  % default \cite such as {\LaTeX} authors are used to

% Where \includegraphics should look for figures
\graphicspath{{./figs/}}
\usepackage{epstopdf}
\DeclareGraphicsExtensions{.eps,.png,.jpg,.pdf}

% Shortcuts
\newcommand{\myhref}[2]{\href{#1}{\textcolor{Blue}{#2}}}
\newcommand{\subitem}[1]{\begin{itemize}[<.->]\item #1 \end{itemize}}
\newcommand{\ghead}[1]{{\tiny #1\\}}
\newcommand{\doi}[1]{\myhref{http://dx.doi.org/#1}{doi:#1}}
\newcommand{\csym}[1]{\textcolor{Blue}{\texttt{#1}}}


%%%%%%%%%%%%%%%%%%%%%%%%%%%%%%%%%%%%%%%%%%%%%%%%%%%%%%%%%%%%%%%%%%%%%%
\title{Possibilities for SED-ML}
\author[Jonathan Cooper]{Jonathan Cooper \and Gary Mirams \and James Osborne}
\institute[University of Oxford]
{Computational Biology Group\\
 Department of Computer Science\\
 University of Oxford}
\date{August 15, 2012}

\begin{document}

\begin{comment}
\begin{abstract}
We have been working on defining and using virtual experiments within a variety of contexts, including cardiac electrophysiology, multi-cellular tissue dynamics, immunology, and synthetic biology.  In doing so we have found that extensions are required if SED-ML is to address our needs.  This talk will introduce some of our use cases, highlight our functional curation framework (\url{https://chaste.cs.ox.ac.uk/cgi-bin/trac.cgi/wiki/FunctionalCuration}) for running experiments on a range of models, and outline our proposals for SED-ML.
\end{abstract}
\end{comment}

\begin{frame}
\titlepage
\end{frame}

%%%%%%%%%%%%%%%%%%%%%%%%%%%%%%%%%%%%%%%%%%%%%%%%%%%%%%%%%%%%%%%%%%%%%%

\begin{frame}{Outline}
\setcounter{tocdepth}{1}
\tableofcontents
\end{frame}


%%%%%%%%%%%%%%%%%%%%%%%%%%%%%%%%%%%%%%%%%%%%%%%%%%%%%%%%%%%%%%%%%%%%%%
\section[Introduction]{Introduction}
\subsection*{Main}
%%%%%%%%%%%%%%%%%%%%%%%%%%%%%%%%%%%%%%%%%%%%%%%%%%%%%%%%%%%%%%%%%%%%%%

\begin{frame}{What problem are we trying to solve?}
\begin{itemize}[<+->]
\item How can the whole modelling process be improved?
\item How can models be reused and composed reliably?
\item Especially as models increase in size and complexity
% Introduce cardiac application.  Make point that even cardiac cell model composed of many units at multiple levels: cell, ion channel, protein subunit.  Reaction networks also build complexity from small repeated concepts.
\end{itemize}
\begin{itemize}[<+->]
\item We want \ldots
  \begin{itemize}
  \item<.-> A single model description that can be used/analysed in multiple ways
  \item Separation of model structure (i.e.\ the equations describing biological function) and experimental scenario
  \item A uniform approach to model fitting, simulation, comparison and validation
  \end{itemize}
\end{itemize}
\end{frame}


\begin{frame}{Models, experiments and protocols}
\begin{itemize}
\item A \alert{model} is a purposeful simplification of reality
\item An \alert{experiment} is the process of stimulating a system to elicit observable responses
\item A \alert{protocol} is a detailed specification for carrying out an experiment
  \subitem{Environmental conditions / parameters, interventions, recordings, filtering \& post-processing, numerical algorithms, etc.}
\end{itemize}
\end{frame}


%%%%%%%%%%%%%%%%%%%%%%%%%%%%%%%%%%%%%%%%%%%%%%%%%%%%%%%%%%%%%%%%%%%%%%
\section[Use cases]{Use cases}
\subsection*{Main}
%%%%%%%%%%%%%%%%%%%%%%%%%%%%%%%%%%%%%%%%%%%%%%%%%%%%%%%%%%%%%%%%%%%%%%

\begin{frame}{Use cases}
\begin{itemize}
\item Functional curation of cardiac cell models
\item Parameter sweeping models of multi-cellular tissue dynamics
\item Experiments in immunology and synthetic biology
 % Won't talk about this in detail, just mention
\end{itemize}
\end{frame}


%%%%%%%%%%%%%%%%%%%%%%%%%%%%%%%%%%%%%%%%%%%%%%%%%%%%%%%%%%%%%%%%%%%%%%
\section{Proposals for SED-ML}
\subsection*{Main}
%%%%%%%%%%%%%%%%%%%%%%%%%%%%%%%%%%%%%%%%%%%%%%%%%%%%%%%%%%%%%%%%%%%%%%



%%%%%%%%%%%%%%%%%%%%%%%%%%%%%%%%%%%%%%%%%%%%%%%%%%%%%%%%%%%%%%%%%%%%%%
\begin{frame}{End matter}
\url{https://chaste.cs.ox.ac.uk/cgi-bin/trac.cgi/wiki/FunctionalCuration}
\end{frame}

\end{document}
