\documentclass[t,xcolor={usenames,dvipsnames}]{beamer}

\mode<presentation>
{
\usetheme{Frankfurt}%{Warsaw}
%\setbeamercovered{transparent}
%\setbeamercolor{background canvas}{bg=white}
}

% Delete these, if you do not want the table of contents to pop up at
% the beginning of each (sub)section:
%\AtBeginSubsection[]
%{
%  \begin{frame}<beamer>{Outline}
%    \tableofcontents[currentsection,currentsubsection]
%  \end{frame}
%}
%\AtBeginSection[]
%{
%  \begin{frame}<beamer>{Outline}
%    \tableofcontents[currentsection]
%  \end{frame}
%}

\usepackage[english]{babel}
\usepackage[latin1]{inputenc}
\usepackage{times}
\usepackage[T1]{fontenc}
\usepackage{verbatim}
\usepackage{url}
\usepackage{amsmath,amssymb}
\usepackage{comment}
\usepackage[overlay,absolute]{textpos}
\usepackage{hyperref}

% Author-date citations
\usepackage[authoryear,round]{natbib}
\let\cite=\citep  % default \cite such as {\LaTeX} authors are used to

% Where \includegraphics should look for figures
\graphicspath{{./figs/}}
\usepackage{epstopdf}
\DeclareGraphicsExtensions{.eps,.png,.jpg,.pdf}

% Shortcuts
\newcommand{\myhref}[2]{\href{#1}{\textcolor{Blue}{#2}}}
\newcommand{\subitem}[1]{\begin{itemize}[<.->]\item #1 \end{itemize}}
\newcommand{\ghead}[1]{{\tiny #1\\}}
\newcommand{\doi}[1]{\myhref{http://dx.doi.org/#1}{doi:#1}}
\newcommand{\csym}[1]{\textcolor{Blue}{\texttt{#1}}}


%%%%%%%%%%%%%%%%%%%%%%%%%%%%%%%%%%%%%%%%%%%%%%%%%%%%%%%%%%%%%%%%%%%%%%
\title{Further Adventures in Functional Curation}
\author{Jonathan Cooper}
\institute[University of Oxford]
{Computational Biology Group\\
 Department of Computer Science\\
 University of Oxford}
\date{January 15, 2013}

\begin{document}

\begin{frame}
\titlepage
\end{frame}

\begin{comment}
Ideas:
 - overview the goals (again), based on more recent material (VPH2012 talk, UL application!)
 - talk about textual syntax for usability, get feedback
 - show the cell-based example (briefly)
 - mention work done with John W in separate project, how this could be an alternative front-end
 - talk about future plans, Erich, parameter sweeping and SED-ML, etc.
 - brief plug / helper request for DTC SWC workshop
\end{comment}

%%%%%%%%%%%%%%%%%%%%%%%%%%%%%%%%%%%%%%%%%%%%%%%%%%%%%%%%%%%%%%%%%%%%%%

\begin{frame}{Outline}
\setcounter{tocdepth}{1}
\tableofcontents
\end{frame}

%%%%%%%%%%%%%%%%%%%%%%%%%%%%%%%%%%%%%%%%%%%%%%%%%%%%%%%%%%%%%%%%%%%%%%
\section{Introduction to functional curation}
\subsection*{Main}
%%%%%%%%%%%%%%%%%%%%%%%%%%%%%%%%%%%%%%%%%%%%%%%%%%%%%%%%%%%%%%%%%%%%%%

\begin{frame}{Setting the context}
\begin{itemize}
\item Model reuse \& simulation result reproducibility
\item Improving the modelling process
\item Confronting models with data
\end{itemize}
\end{frame}


\begin{frame}{Setting the context}
\begin{itemize}
\item Many models are available in languages such as CellML
\item But, how can we\ldots
  \begin{itemize}
  \item Determine a model's functionality, i.e.\ its suitability or limitations for a new study?
  \item Re-use a model in a different experimental context?
  \item Compare different models' behaviours under the same experiment?  (Compare hypotheses.)
  \end{itemize}
\end{itemize}
\end{frame}


\begin{frame}{Goal of this work}
\begin{itemize}
\item Separate \alert{model structure} and \alert{experimental scenario}
\item Provide a framework for a coherent approach to model fitting, simulation, comparison and validation
  \subitem{Continuous evaluation of model predictions against experimental data, throughout model lifecycle}
\end{itemize}
\begin{center}
\includegraphics[width=\textwidth]{VirtEx_overview}
\end{center}
\end{frame}


\begin{frame}{Foundations: a mosaic of standards}
\begin{center}
\includegraphics<1->[scale=.5]{standards_mosaic}\\
{\tiny Fig.: Mosaic of standards, adapted from (\textit{Chelliah et al., 2009, DILS})}
\end{center}
\end{frame}

\begin{frame}{Foundations: a mosaic of standards}
\begin{center}
\includegraphics<1->[scale=.5]{standards_mosaic}\\
{\tiny Fig.: Mosaic of standards, adapted from (\textit{Chelliah et al., 2009, DILS})}
\end{center}
\TPGrid{2}{2}
\begin{textblock}{2}[0.5,0.5](1,1.2)
\centering
\includegraphics[scale=.8]{COMBINE}
\end{textblock}
\end{frame}


\begin{frame}{Simulation Experiment Description Markup Language}
\begin{center}
\includegraphics[scale=.5]{SEDML_overview}\\
{\tiny Fig.: SED-ML structure. \textit{Waltemath et al., BMC Sys Biol (2011)}}
\end{center}
\end{frame}

%%%%%%%%%%%%%%%%%%%%%%%%%%%%%%%%%%%%%%%%%%%%%%%%%%%%%%%%%%%%%%%%%%%%%%
\section{Protocol language details}
\subsection*{Main}
%%%%%%%%%%%%%%%%%%%%%%%%%%%%%%%%%%%%%%%%%%%%%%%%%%%%%%%%%%%%%%%%%%%%%%

\begin{frame}{What goes in a protocol?}
\begin{itemize}
\item Definition of the interface with the model being experimented on
\item Definition of simulations to perform
\item Post-processing operations on simulation results
\item Description of what to plot
\item ``Extra bits''
  \begin{itemize}
  \item Inputs and outputs of the whole protocol
  \item Library of functions and/or variables
  \item Imports of other protocols
  \end{itemize}
\end{itemize}
\end{frame}

\begin{frame}{Protocol structure}
\begin{center}
\includegraphics[width=\textwidth]{protocol_language}
\end{center}
\end{frame}

%%%%%%%%%%%%%%%%%%%%%%%%%%%%%%%%%%%%%%%%%%%%%%%%%%%%%%%%%%%%
\subsection{Interfacing protocols and models}

\begin{frame}{Interfacing protocols and models}
\begin{itemize}
\item Challenges arise from desire to apply a single experiment description to any model
  \subitem{And need to compare results for different models}
\item There are variations in modelling conventions
  \begin{itemize}
  \item Different names for the same `thing'
  \item Different ways of representing biology in mathematics
  \end{itemize}
\item An experiment may isolate a sub-component of the model (e.g.\ voltage clamp)
\end{itemize}
\end{frame}

\begin{frame}{Referring to model variables}
\begin{itemize}
\item How do we cope with different variable names for the same entity?
  \begin{itemize}
  \item Transmembrane potential: $V$, $V_m$, $E$
  \item Stimulus current: $i_{\mathrm{Stim}}$, $I_{\mathrm{st}}$, $i_{\mathrm{pulse}}$, $i_{\mathrm{ext}}$
  \item Membrane capacitance: $C$, $C_m$, $\mathit{Acap}$
  \end{itemize}
\item Use \alert{ontological annotation} of variables
%\item Protocol can use \texttt{prefix:name} notation as for XML namespaces
\item No need for `approved' ontology --- just need model \& protocol to agree
\end{itemize}
\end{frame}

\begin{frame}{Units conversions}
\begin{itemize}
\item Different models use different units
\item Protocol declares the units it uses, and conversions applied automatically
  \note{This is easy for scalings within a dimension, but models can use different approaches to representing the same biology, e.g.\ different normalisation for ionic currents}
\item ``Biology-aware'' conversion rules can be defined
  \begin{itemize}
  \item A unary function for converting a value from one dimension to another
  \item Can refer to model variables using ontology terms
  \item Fall-back to next rule if required variables don't exist
  \item See also \doi{10.1016/j.pbiomolbio.2011.06.002}
  \end{itemize}
\end{itemize}
\end{frame}

\begin{frame}{Model modifications}
\begin{itemize}
\item A model is viewed as a \alert{system of equations}, independent of modelling language
\item<2-> Protocol specifies which variables are \alert{inputs} and \alert{outputs}
\item<2-> Inputs become parameters that can be set by the protocol
  \subitem{e.g.\ voltage clamp experiment}
\item<2-> Only those equations required for the given outputs need be computed
\item<3-> Equations may also be \alert{added or replaced}
  \subitem{e.g.\ to specify a stimulus current waveform}
\end{itemize}
\end{frame}

%\begin{frame}{Interfacing protocols and models}
%This will probably move, if we use it at all.
%\begin{center}
%\includegraphics[width=.9\textwidth]{schematic_v4}
%\end{center}
%\end{frame}

%%%%%%%%%%%%%%%%%%%%%%%%%%%%%%%%%%%%%%%%%%%%%%%%%%%%%%%%%%%%
\subsection{Defining simulations}

\begin{frame}{Sequenced and nested simulations}
\begin{itemize}[<+->]
\item Basic simulation: timecourse solve of model equations
\item An experiment may have `setup' and `measurement' phases $\implies$ simulations should be able to run in sequence
\item \alert{Nesting} simulations supports parameter scans, repeated runs, distribution sampling, etc.
  \subitem{\alert{Model outputs therefore become regular $n$-dimensional arrays}}
\end{itemize}
\end{frame}

\begin{frame}{Simulation loops}
\begin{itemize}
\item Each simulation requires a \alert{range} over which to iterate for generating output points
\item These are given a name for the loop variable, and its units
  \begin{description}
  \item[UniformStepper] has start, stride, and end parameters
  \item[VectorStepper] allows irregular steps, using post-processing language constructs to define an array of values for the loop variable
  \item[WhileStepper] enables an outer loop to continue as long as a condition is met
  \end{description}
\end{itemize}
\end{frame}

\begin{frame}{Modifiers}
\begin{itemize}
\item Each simulation can also have a collection of \alert{modifiers}
  \begin{description}
  \item[SaveState] store the current model state, giving it a name
  \item[ResetState] reset the model to a stored state or initial conditions
  \item[SetVariable] set the value of a model variable\\
      The value is given by an expression in the post-processing language,
      and can access the current range value for this or any outer loop.
  \end{description}
\item Each can be applied just at the start or end of a simulation, or prior to each loop
\end{itemize}
\end{frame}

%%%%%%%%%%%%%%%%%%%%%%%%%%%%%%%%%%%%%%%%%%%%%%%%%%%%%%%%%%%%
\subsection{Post-processing}

\begin{frame}{Post-processing}
\begin{itemize}
\item Raw simulation results are rarely what we're after
\item Post-processing required to obtain markers of interest
\item For a protocol exchange language, we need something more restrictive (and easier to implement) than general Matlab or similar
\item But we still need the ability to describe complex operations
  \subitem{e.g.\ maximum, average, APD, \ldots}
\end{itemize}
\end{frame}

\begin{frame}{Post-processing language}
\begin{itemize}
\item For minimal implementation overhead, based on MathML, with as few as possible added features
\item Key additions:
  \begin{itemize}
  \item Operators for working with $n$-dimensional arrays
  \item Sequencing operations (defining variables, assertions)
  \item Defining functions (that can also be passed to other functions)
  \end{itemize}
\item Not just used for post-processing: also input specifications, library definitions, etc.
\item Technically, this is a \alert{pure} functional $n$-dimensional array based programming language
\end{itemize}
\end{frame}

\begin{frame}{Main special expressions}
\note{Go through these in more detail on whiteboard!}
\footnotesize
\begin{description}
\item[\csym{newArray}] Create a new array
  \begin{itemize}
  \item by listing elements (which may be arrays)
  \item by \alert{comprehension} using a generator expression with index ranges% (abusing \texttt{domainofapplication})
  \end{itemize}
\item[\csym{view}] Extract a sub-array
  \subitem{Can use arbitrary (even negative) strides over any dimension, with wildcards}
\item[\csym{map}] Map an $n$-ary function onto $n$ arrays element-wise
\item[\csym{fold}] Collapse an array along a single dimension using a binary function
  \subitem{Used to define \texttt{sum}, \texttt{max}, etc.}
\item[\csym{find}] Find indices where the operand array is non-zero
\item[\csym{index}] Create a sub-array containing only the given indices
  \subitem{Various options for avoiding irregular results}
\end{description}
\end{frame}

\begin{frame}{Miscellaneous extra features}
\begin{itemize}
\item Accessor operation for testing values (IS\_ARRAY, SHAPE, etc.)
\item Functions can have default parameters, and the default can be asked for explicitly
\item Can define tuples (pairs, triples, etc.), e.g.\ to return multiple values
\item Location information for user-friendly error messages
\end{itemize}
\end{frame}

\begin{frame}{Example: APD}
\begin{itemize}
\item Find the maximum upstroke velocity locations to locate APs
  \subitem{Localised max above threshold}
\item Work with extended arrays: extra dimension of size `max num APs'
\item Find the peaks: First \texttt{t} s.t.\ \texttt{V == local\_Vmax \&\& t > t\_up}
\item Assume resting potential is global minimum, and calculate relaxation potential
\item Find (with interpolation) the first time before and after each peak where the relaxation potential is crossed
\item APDs are differences between these arrays
\item DIs = \texttt{diff(t\_ap\_starts) - apds[:-1]}
\end{itemize}
\end{frame}

%%%%%%%%%%%%%%%%%%%%%%%%%%%%%%%%%%%%%%%%%%%%%%%%%%%%%%%%%%%%
\subsection{Other protocol features}

\begin{frame}{Nested protocols}
\begin{itemize}
\item Since a protocol has inputs and outputs, it can be viewed as a kind of model
\item The ``system of equations'' abstraction does not apply, so model modifications are not possible
\item This does effectively allow us to interleave post-processing and simulation however
\item So we can do e.g.\ dynamic restitution without breaking the `regular $n$-d array' data model
\end{itemize}
\end{frame}

\begin{frame}{Importing protocols}
\begin{itemize}
\item Define libraries of common protocol components
  \begin{itemize}
  \item Especially useful for post-processing functions
  \item Can also define standard simulations, model changes, etc.
  \end{itemize}
\item Specialise generic protocol for particular scenario
  \subitem{e.g.\ set basic cycle length for species-specific S1-S2}
\end{itemize}
\end{frame}

%%%%%%%%%%%%%%%%%%%%%%%%%%%%%%%%%%%%%%%%%%%%%%%%%%%%%%%%%%%%%%%%%%%%%%
\section{Conclusions and future directions}
\subsection*{Main}
%%%%%%%%%%%%%%%%%%%%%%%%%%%%%%%%%%%%%%%%%%%%%%%%%%%%%%%%%%%%%%%%%%%%%%

\begin{frame}{Challenge the speaker}
I think the language now has all the features required for cardiac cell electrophysiology.

Describe your favourite protocol, and I'll try to outline how it would be encoded.
\end{frame}

\begin{frame}{Limitations of current implementation}
\begin{itemize}
\item System implemented as Chaste user project
\item It's still a prototype!
\item Usability is limited
  \begin{itemize}
  \item Protocol language is XML, lacking nice front-end
  \item Especially difficult to write good post-processing functions (see next slide)
  \end{itemize}
\item Performance improvements and parallelisation needed for large protocols
\end{itemize}
\end{frame}

\begin{frame}{Limitations in APD calculation}
\begin{center}
\includegraphics<1>{faber_rudy_2000_s1s2_curve}
\only<2>{\vspace{-.5cm}}
\includegraphics<2>[height=.9\textheight]{faber_rudy_2000_COR_comparison}
\end{center}
\only<3>{\vspace{-1.5cm}Nygren 1998 (atrial model)}
\includegraphics<3>[height=.8\textheight]{nygren_s1s2_V}
\end{frame}

\begin{frame}{Conclusions and plans}
\begin{itemize}
\item Functional curation offers great scope for transforming our approach to modelling
\item Initial prototype is promising
\item Lots to do to demonstrate this fully
  \begin{itemize}
  \item (With Gary) Explore more cardiac protocols
  \item (With OCISB) Apply to cell-cycle models
  \item (With MSRC) Apply to immunology and synthetic biology
  \item (With Ozzy) Apply to cell-based Chaste simulations
  \item (With Mark) Work on interface and data storage
  \item (With ?) Improve performance
  \end{itemize}
\end{itemize}
\end{frame}


%%%%%%%%%%%%%%%%%%%%%%%%%%%%%%%%%%%%%%%%%%%%%%%%%%%%%%%%%%%%%%%%%%%%%%
\begin{frame}{Acknowledgments}
Gary Mirams\\
Chaste team\\
Alan Garny, Steven Niederer

Reference publication: \doi{10.1016/j.pbiomolbio.2011.06.003}

\begin{center}
\includegraphics[scale=.9]{chaste-266x60}\\ \vspace{.4cm}
\includegraphics[width=.55\textwidth]{EPSRC1RGBLO} \hspace{.1cm}
\includegraphics[scale=.55]{logo_msr}
% TODO: Add 2020 Science
\end{center}
\end{frame}

\end{document}
